\chapter{Parsing and Building Messages}

% ============================================================================
\section{Parsing messages}

\subsection{Introduction} % --------------------------------------------------

Parsing is the process of creating a structured representation (for example,
a hierarchy of C++ objects) of a message from its ``textual'' representation
(the raw data that is actually sent on the Internet).

For example, say you have the following email in a file called "hello.eml":

\begin{verbatim}
Date: Thu, Oct 13 2005 15:22:46 +0200
From: Vincent <vincent@vmime.org>
To: you@vmime.org
Subject: Hello from VMime!

A simple message to test VMime
\end{verbatim}

The following code snippet shows how you can easily obtain a
{\vcode vmime::message} object from data in this file:

\begin{lstlisting}[caption={Parsing a message from a file}]
// Read data from file
std::ifstream file;
file.open("hello.eml", std::ios::in | std::ios::binary);

vmime::utility::inputStreamAdapter is(file);

vmime::string data;
vmime::utility::outputStreamStringAdapter os(data);

vmime::utility::bufferedStreamCopy(is, os);

// Actually parse the message
vmime::shared_ptr <vmime::message> msg = vmime::make_shared <vmime::message>();
msg->parse(data);

vmime::shared_ptr <vmime::header> hdr = msg->getHeader();
vmime::shared_ptr <vmime::body> bdy = msg->getBody();

// Now, you can extract some of its components
vmime::charset ch(vmime::charsets::UTF_8);

std::cout
   << "The subject of the message is: "
   << hdr->Subject()->getValue <vmime::text>()->getConvertedText(ch)
   << std::endl
   << "It was sent by: "
   << hdr->From()->getValue <vmime::mailbox>()->getName().getConvertedText(ch)
   << " (email: " << hdr->From()->getValue <vmime::mailbox>()->getEmail() << ")"
   << std::endl;
\end{lstlisting}

The output of this program is:

\begin{verbatim}
The subject of the message is: Hello from VMime!
It was sent by: Vincent (email: vincent@vmime.org)
\end{verbatim}


\subsection{Using the {\vcode vmime::messageParser} object} % ----------------

The {\vcode vmime::messageParser} object allows to parse messages in a more
simple manner. You can obtain all the text parts and attachments as well as
basic fields (expeditor, recipients, subject...), without dealing with
MIME message structure.

\begin{lstlisting}[caption={Using {\vcode vmime::messageParser} to parse
more complex messages}]
// Read data from file
std::ifstream file;
file.open("hello.eml", std::ios::in | std::ios::binary);

vmime::utility::inputStreamAdapter is(file);

vmime::string data;
vmime::utility::outputStreamStringAdapter os(data);

vmime::utility::bufferedStreamCopy(is, os);

// Actually parse the message
vmime::shared_ptr <vmime::message> msg = vmime::make_shared <vmime::message>();
msg->parse(data);

// Here start the differences with the previous example
vmime::messageParser mp(msg);

// Output information about attachments
std::cout << "Message has " << mp.getAttachmentCount()
   << " attachment(s)" << std::endl;

for (int i = 0 ; i < mp.getAttachmentCount() ; ++i)
{
   vmime::shared_ptr <const vmime::attachment> att = mp.getAttachmentAt(i);
   std::cout << "   - " << att->getType().generate() << std::endl;
}

// Output information about text parts
std::cout << "Message has " << mp.getTextPartCount()
   << " text part(s)" << std::endl;

for (int i = 0 ; i < mp.getTextPartCount() ; ++i)
{
   vmime::shared_ptr <const vmime::textPart> tp = mp.getTextPartAt(i);

   // text/html
   if (tp->getType().getSubType() == vmime::mediaTypes::TEXT_HTML)
   {
      vmime::shared_ptr <const vmime::htmlTextPart> htp =
         vmime::dynamicCast <const vmime::htmlTextPart>(tp);

      // HTML text is in tp->getText()
      // Plain text is in tp->getPlainText()

      // Enumerate embedded objects
      for (int j = 0 ; j < htp->getObjectCount() ; ++j)
      {
         vmime::shared_ptr <const vmime::htmlTextPart::embeddedObject> obj =
            htp->getObjectAt(j);

         // Identifier (Content-Id or Content-Location) is obj->getId()
         // Object data is in obj->getData()
      }
   }
   // text/plain or anything else
   else
   {
      // Text is in tp->getText()
   }
}
\end{lstlisting}


% ============================================================================
\section{Building messages}

\subsection{A simple message\label{msg-building-simple-message}} % -----------

Of course, you can build a MIME message from scratch by creating the various
objects that compose it (parts, fields, etc.). The following is an example of
how to achieve it:

\begin{lstlisting}[caption={Building a simple message from scratch}]
vmime::shared_ptr <vmime::message> msg = vmime::make_shared <vmime::message>();

vmime::shared_ptr <vmime::header> hdr = msg->getHeader();
vmime::shared_ptr <vmime::body> bdy = msg->getBody();

vmime::shared_ptr <vmime::headerFieldFactory> hfFactory =
   vmime::headerFieldFactory::getInstance();

// Append a 'Date:' field
vmime::shared_ptr <vmime::headerField> dateField =
   hfFactory->create(vmime::fields::DATE);

dateField->setValue(vmime::datetime::now());
hdr->appendField(dateField);

// Append a 'Subject:' field
vmime::shared_ptr <vmime::headerField> subjectField =
   hfFactory->create(vmime::fields::SUBJECT);

subjectField->setValue(vmime::text("Message subject"));
hdr->appendField(subjectField);

// Append a 'From:' field
vmime::shared_ptr <vmime::headerField> fromField =
   hfFactory->create(vmime::fields::FROM);

fromField->setValue
   (vmime::make_shared <vmime::mailbox>("me@vmime.org"));
hdr->appendField(fromField);

// Append a 'To:' field
vmime::shared_ptr <vmime::headerField> toField =
   hfFactory->create(vmime::fields::TO);

vmime::shared_ptr <vmime::mailboxList> recipients =
   vmime::make_shared <vmime::mailboxList>();

recipients->appendMailbox
   (vmime::make_shared <vmime::mailbox>("you@vmime.org"));

toField->setValue(recipients);
hdr->appendField(toField);

// Set the body contents
bdy->setContents(vmime::make_shared <vmime::stringContentHandler>
   ("This is the text of your message..."));

// Output raw message data to standard output
vmime::utility::outputStreamAdapter out(std::cout);
msg->generate(out);
\end{lstlisting}

As you can see, this is a little fastidious. Hopefully, VMime also offers a
more simple way for creating messages. The {\vcode vmime::messageBuilder}
object can create basic messages that you can then customize.

The following code can be used to build exactly the same message as in the
previous example, using the {\vcode vmime::messageBuilder} object:

\begin{lstlisting}[caption={Building a simple message
using {\vcode vmime::messageBuilder}}]
try
{
   vmime::messageBuilder mb;

   // Fill in some header fields and message body
   mb.setSubject(vmime::text("Message subject"));
   mb.setExpeditor(vmime::mailbox("me@vmime.org"));
   mb.getRecipients().appendAddress
      (vmime::make_shared <vmime::mailbox>("you@vmime.org"));

   mb.getTextPart()->setCharset(vmime::charsets::ISO8859_15);
   mb.getTextPart()->setText(vmime::make_shared <vmime::stringContentHandler>
   	("This is the text of your message..."));

   // Message construction
   vmime::shared_ptr <vmime::message> msg = mb.construct();

   // Output raw message data to standard output
   vmime::utility::outputStreamAdapter out(std::cout);
   msg->generate(out);
}
// VMime exception
catch (vmime::exception& e)
{
   std::cerr << "vmime::exception: " << e.what() << std::endl;
}
// Standard exception
catch (std::exception& e)
{
   std::cerr << "std::exception: " << e.what() << std::endl;
}
\end{lstlisting}


\subsection{Adding an attachment} % ------------------------------------------

Dealing with attachments is quite simple. Add the following code to the
previous example to attach a file to the message:

\begin{lstlisting}[caption={Building a message with an attachment using
{\vcode vmime::messageBuilder}}]
// Create an attachment
vmime::shared_ptr <vmime::fileAttachment> att =
   vmime::make_shared <vmime::fileAttachment>
   (
     /* full path to file */  "/home/vincent/paris.jpg",
     /* content type */       vmime::mediaType("image/jpeg),
     /* description */        vmime::text("My holidays in Paris")
   );

// You can also set some infos about the file
att->getFileInfo().setFilename("paris.jpg");
att->getFileInfo().setCreationDate
   (vmime::datetime("30 Apr 2003 14:30:00 +0200"));

// Add this attachment to the message
mb.appendAttachment(att);
\end{lstlisting}


\subsection{HTML messages and embedded objects} % ----------------------------

VMime also supports aggregate messages, which permits to build MIME messages
containing HTML text and embedded objects (such as images). For more information
about aggregate messages, please read RFC-2557 (\emph{MIME Encapsulation of
Aggregate Documents, such as HTML}).

Creating such messages is quite easy, using the {\vcode vmime::messageBuilder}
object. The following code constructs a message containing text in both plain
and HTML format, and a JPEG image:

\begin{lstlisting}[caption={Building an HTML message with an embedded image
using the {\vcode vmime::messageBuilder}}]
// Fill in some header fields
mb.setSubject(vmime::text("An HTML message"));
mb.setExpeditor(vmime::mailbox("me@vmime.org"));
mb.getRecipients().appendAddress
   (vmime::make_shared <vmime::mailbox>("you@vmime.org"));

// Set the content-type to "text/html": a text part factory must be
// available for the type you are using. The following code will make
// the message builder construct the two text parts.
mb.constructTextPart(vmime::mediaType
   (vmime::mediaTypes::TEXT, vmime::mediaTypes::TEXT_HTML));

// Set contents of the text parts; the message is available in two formats:
// HTML and plain text. The HTML format also includes an embedded image.
vmime::shared_ptr <vmime::htmlTextPart> textPart =
   vmime::dynamicCast <vmime::htmlTextPart>(mb.getTextPart());

// -- Add the JPEG image (the returned identifier is used to identify the
// -- embedded object in the HTML text, the famous "CID", or "Content-Id").
// -- Note: you can also read data from a file; see the next example.
const vmime::string id = textPart->addObject("<...image data...>",
   vmime::mediaType(vmime::mediaTypes::IMAGE, vmime::mediaTypes::IMAGE_JPEG));

// -- Set the text
textPart->setCharset(vmime::charsets::ISO8859_15);

textPart->setText(vmime::make_shared <vmime::stringContentHandler>
   ("This is the <b>HTML text</b>, and the image:<br/>"
    "<img src=\"") + id + vmime::string("\"/>"));

textPart->setPlainText(vmime::make_shared <vmime::stringContentHandler>
   ("This is the plain text."));
\end{lstlisting}

This will create a message having the following structure:

\begin{verbatim}
multipart/alternative
   text/plain
   multipart/related
      text/html
      image/jpeg
\end{verbatim}

You can easily tell VMime to read the embedded object data from a file. The
following code opens the file \emph{/path/to/image.jpg}, connects it to an
input stream, then add an embedded object:

\begin{lstlisting}
vmime::utility::fileSystemFactory* fs =
   vmime::platform::getHandler()->getFileSystemFactory();

vmime::shared_ptr <vmime::utility::file> imageFile =
   fs->create(fs->stringToPath("/path/to/image.jpg"));

vmime::shared_ptr <vmime::contentHandler> imageCts =
   vmime::make_shared <vmime::streamContentHandler>
      (imageFile->getFileReader()->getInputStream(), imageFile->getLength());

const vmime::string cid = textPart.addObject(imageCts,
   vmime::mediaType(vmime::mediaTypes::IMAGE, vmime::mediaTypes::IMAGE_JPEG));
\end{lstlisting}


% ============================================================================
\section{Working with attachments: the attachment helper}

The {\vcode attachmentHelper} object allows listing all attachments in a
message, as well as adding new attachments, without using the
{\vcode messageParser} and {\vcode messageBuilders} objects. It can work
directly on messages and body parts.

To use it, you do not need any knowledge about how attachment parts should
be organized in a MIME message.

The following code snippet tests if a body part is an attachment, and if so,
extract its contents to the standard output:

\begin{lstlisting}[caption={Testing if a body part is an attachment}]
vmime::shared_ptr <vmime::bodyPart> part;  // suppose we have a body part

if (vmime::attachmentHelper::isBodyPartAnAttachment(part))
{
   // The body part contains an attachment, get it
   vmime::shared_ptr <const vmime::attachment> attach =
      attachmentHelper::getBodyPartAttachment(part);

   // Extract attachment data to standard output
   vmime::utility::outputStreamAdapter out(std::cout);
   attach->getData()->extract(out);
}
\end{lstlisting}

You can also easily extract all attachments from a message:

\begin{lstlisting}[caption={Extracting all attachments from a message}]
vmime::shared_ptr <vmime::message> msg;  // suppose we have a message

const std::vector <ref <const attachment> > atts =
   attachmentHelper::findAttachmentsInMessage(msg);
\end{lstlisting}

Finally, the {\vcode attachmentHelper} object can be used to add an
attachment to an existing message, whatever it contains (text parts,
attachments, ...). The algorithm can modify the structure of the
message if needed (eg. add a \emph{multipart/mixed} part if no one
exists in the message). Simply call the {\vcode addAttachment}
function:

\begin{lstlisting}[caption={Adding an attachment to an existing message}]
vmime::shared_ptr <vmime::message> msg;  // suppose we have a message

// Create an attachment
vmime::shared_ptr <vmime::fileAttachment> att =
   vmime::make_shared <vmime::fileAttachment>
   (
     /* full path to file */  "/home/vincent/paris.jpg",
     /* content type */       vmime::mediaType("image/jpeg),
     /* description */        vmime::text("My holidays in Paris")
   );

// Attach it to the message
vmime::attachmentHelper::addAttachment(msg, att);
\end{lstlisting}

